\documentclass[a4paper]{article}
\usepackage{ctex}
\usepackage[affil-it]{authblk}
\usepackage[backend=bibtex,style=numeric]{biblatex}
\usepackage{amsmath,amsthm,amssymb,amsfonts}
\usepackage{mathrsfs}
\usepackage{bm}
\usepackage{graphicx}
\usepackage{geometry}

\geometry{margin=1.5cm, vmargin={0pt,1cm}}
\setlength{\topmargin}{-1cm}
\setlength{\paperheight}{29.7cm}
\setlength{\textheight}{25.3cm}

\addbibresource{citation.bib}

\newcommand{\Rmnum}[1]{\uppercase\expandafter{\romannumeral #1}}  
\newcommand{\rmnum}[1]{\romannumeral #1}

\begin{document}
% =================================================
\title{FDTD 求解器}

\author{zsh945
  \thanks{邮箱: \texttt{3097714673@qq.com}}}

\date{更新时间: \today}

\maketitle

% ============================================
\section{Yee 网格\cite{jin_theory_2015}}

\par 一般的数值微分算法拓展到三维问题会遇到一些严重的问题。
\begin{itemize}
    \item 当网格点位于两种不同媒质的分界面上时,为了保证切向场的连续和法向场的不连续,
    需要进行相当烦琐的处理。
    \item 当一个网格点位于导体或媒质的边缘
    或拐角时,这一点的法向没有明确的定义,而其场分量可能是无限大的,即奇异的,因此这
    个网格点处的场值无法精确描述。
\end{itemize}
\par 而 Yee 网格作为一种独特的离散方法,成功地解决了这些问题。
考虑时域 Maxwell 方程组:
\begin{align}
    \label{eq:Yee 三维 Maxwell 方程组-矢量形式-1}
    \nabla \times \bm{\mathcal{E}} &= -\mu \frac{\partial \bm{\mathcal{H}}}{\partial t}\\
    \label{eq:Yee 三维 Maxwell 方程组-矢量形式-2}
    \nabla \times \bm{\mathcal{H}} &= 
    \varepsilon \frac{\partial \bm{\mathcal{E}}}{\partial t}
    +\sigma \bm{\mathcal{E}}
    +\bm{\mathcal{J}}_i
\end{align}
\par 这两个矢量方程可以写成 6 个标量方程
\begin{align}
    \label{eq:Yee 三维 Maxwell 方程组-1}
    \frac{\partial \mathcal{E}_z}{\partial y}
    -\frac{\partial \mathcal{E}_y}{\partial z}
    &= -\mu \frac{\partial \mathcal{H}_x}{\partial t}\\
    \label{eq:Yee 三维 Maxwell 方程组-2}
    \frac{\partial \mathcal{E}_x}{\partial z}
    -\frac{\partial \mathcal{E}_z}{\partial x}
    &= -\mu \frac{\partial \mathcal{H}_y}{\partial t}\\
    \label{eq:Yee 三维 Maxwell 方程组-3}
    \frac{\partial \mathcal{E}_y}{\partial x}
    -\frac{\partial \mathcal{E}_x}{\partial y}
    &= -\mu \frac{\partial \mathcal{H}_z}{\partial t}\\
    \label{eq:Yee 三维 Maxwell 方程组-4}
    \frac{\partial \mathcal{H}_z}{\partial y}
    -\frac{\partial \mathcal{H}_y}{\partial z}
    &= \varepsilon \frac{\partial \mathcal{E}_x}{\partial t}
    +\sigma \mathcal{E}_x + \mathcal{J}_x\\
    \label{eq:Yee 三维 Maxwell 方程组-5}
    \frac{\partial \mathcal{H}_x}{\partial z}
    -\frac{\partial \mathcal{H}_z}{\partial x}
    &= \varepsilon \frac{\partial \mathcal{E}_y}{\partial t}
    +\sigma \mathcal{E}_y + \mathcal{J}_y\\
    \label{eq:Yee 三维 Maxwell 方程组-6}
    \frac{\partial \mathcal{H}_y}{\partial x}
    -\frac{\partial \mathcal{H}_x}{\partial y}
    &= \varepsilon \frac{\partial \mathcal{E}_z}{\partial t}
    +\sigma \mathcal{E}_z + \mathcal{J}_z
\end{align}
\par 如图 \ref{fig:YeeLattice_3D} 所示,在单元每条边的中心位置
对电场分量采样,在单元每个面的中心位置对磁场分量采样。
\begin{figure}[htbp]
    \centering
    \includegraphics[width=0.4\textwidth]{./image/YeeLattice_3D.png}
    \caption{Yee 网格三维示意图}
    \label{fig:YeeLattice_3D}
\end{figure}
\par 对于 Yee 网格,
式 (\ref{eq:Yee 三维 Maxwell 方程组-1})
到式 (\ref{eq:Yee 三维 Maxwell 方程组-6})
的时间步进公式分别为
\begin{equation}
    \begin{aligned}
        \mathcal{H}_x^{n+\frac{1}{2}}\left(i,j+\frac{1}{2},k+\frac{1}{2}\right)=
        &\ \mathcal{H}_x^{n-\frac{1}{2}}\left(i,j+\frac{1}{2},k+\frac{1}{2}\right)\\
        &-\frac{\Delta t}{\mu \Delta y}\left[
            \mathcal{E}_z^n\left(i,j+1,k+\frac{1}{2}\right)
            -\mathcal{E}_z^n\left(i,j,k+\frac{1}{2}\right)
        \right]\\
        &+\frac{\Delta t}{\mu \Delta z}\left[
            \mathcal{E}_y^n\left(i,j+\frac{1}{2},k+1\right)
            -\mathcal{E}_y^n\left(i,j+\frac{1}{2},k\right)
        \right]
    \end{aligned}
    \label{eq:Yee 三维时间步进公式-1}
\end{equation}
\begin{equation}
    \begin{aligned}
        \mathcal{H}_y^{n+\frac{1}{2}}\left(i+\frac{1}{2},j,k+\frac{1}{2}\right)=
        &\ \mathcal{H}_y^{n-\frac{1}{2}}\left(i+\frac{1}{2},j,k+\frac{1}{2}\right)\\
        &-\frac{\Delta t}{\mu \Delta z}\left[
            \mathcal{E}_x^n\left(i+\frac{1}{2},j,k+1\right)
            -\mathcal{E}_x^n\left(i\frac{1}{2},j,k\right)
        \right]\\
        &+\frac{\Delta t}{\mu \Delta x}\left[
            \mathcal{E}_z^n\left(i+1,j,k+\frac{1}{2}\right)
            -\mathcal{E}_z^n\left(i,j,k+\frac{1}{2}\right)
        \right]
    \end{aligned}
    \label{eq:Yee 三维时间步进公式-2}
\end{equation}
\begin{equation}
    \begin{aligned}
        \mathcal{H}_z^{n+\frac{1}{2}}\left(i+\frac{1}{2},j+\frac{1}{2},k\right)=
        &\ \mathcal{H}_z^{n-\frac{1}{2}}\left(i+\frac{1}{2},j+\frac{1}{2},k\right)\\
        &-\frac{\Delta t}{\mu \Delta x}\left[
            \mathcal{E}_y^n\left(i+1,j+\frac{1}{2},k\right)
            -\mathcal{E}_y^n\left(i,j+\frac{1}{2},k\right)
        \right]\\
        &+\frac{\Delta t}{\mu \Delta y}\left[
            \mathcal{E}_x^n\left(i+\frac{1}{2},j+1,k\right)
            -\mathcal{E}_x^n\left(i+\frac{1}{2},j,k\right)
        \right]
    \end{aligned}
    \label{eq:Yee 三维时间步进公式-3}
\end{equation}
\begin{equation}
    \begin{aligned}
        \mathcal{E}_x^{n+1}\left(i+\frac{1}{2},j,k\right)=
        &\ \frac{1}{\beta\left(i+\frac{1}{2},j,k\right)}
        \Bigg\{\alpha\left(i+\frac{1}{2},j,k\right)
        \mathcal{E}_x^n\left(i+\frac{1}{2},j,k\right)\\
        &+\frac{1}{\Delta y}\left[
            \mathcal{H}_z^{n+\frac{1}{2}}\left(i+\frac{1}{2},j+\frac{1}{2},k\right)
            -\mathcal{H}_z^{n+\frac{1}{2}}\left(i+\frac{1}{2},j-\frac{1}{2},k\right)
        \right]\\
        &-\frac{1}{\Delta z}\left[
            \mathcal{H}_y^{n+\frac{1}{2}}\left(i+\frac{1}{2},j,k+\frac{1}{2}\right)
            -\mathcal{H}_y^{n+\frac{1}{2}}\left(i+\frac{1}{2},j,k-\frac{1}{2}\right)
        \right]\\
        &-\mathcal{J}_x^{n+\frac{1}{2}}\left(i+\frac{1}{2},j,k\right)
        \Bigg\}
    \end{aligned}
    \label{eq:Yee 三维时间步进公式-4}
\end{equation}
\begin{equation}
    \begin{aligned}
        \mathcal{E}_y^{n+1}\left(i,j+\frac{1}{2},k\right)=
        &\ \frac{1}{\beta\left(i,j+\frac{1}{2},k\right)}
        \Bigg\{\alpha\left(i,j+\frac{1}{2},k\right)
        \mathcal{E}_y^n\left(i,j+\frac{1}{2},k\right)\\
        &+\frac{1}{\Delta z}\left[
            \mathcal{H}_x^{n+\frac{1}{2}}\left(i,j+\frac{1}{2},k+\frac{1}{2}\right)
            -\mathcal{H}_x^{n+\frac{1}{2}}\left(i,j+\frac{1}{2},k-\frac{1}{2}\right)
        \right]\\
        &-\frac{1}{\Delta x}\left[
            \mathcal{H}_z^{n+\frac{1}{2}}\left(i+\frac{1}{2},j+\frac{1}{2},k\right)
            -\mathcal{H}_z^{n+\frac{1}{2}}\left(i-\frac{1}{2},j+\frac{1}{2},k\right)
        \right]\\
        &-\mathcal{J}_y^{n+\frac{1}{2}}\left(i,j+\frac{1}{2},k\right)
        \Bigg\}
    \end{aligned}
    \label{eq:Yee 三维时间步进公式-5}
\end{equation}
\begin{equation}
    \begin{aligned}
        \mathcal{E}_z^{n+1}\left(i,j,k+\frac{1}{2}\right)=
        &\ \frac{1}{\beta\left(i,j,k+\frac{1}{2}\right)}
        \Bigg\{\alpha\left(i,j,k+\frac{1}{2}\right)
        \mathcal{E}_z^n\left(i,j,k+\frac{1}{2}\right)\\
        &+\frac{1}{\Delta x}\left[
            \mathcal{H}_y^{n+\frac{1}{2}}\left(i+\frac{1}{2},j,k+\frac{1}{2}\right)
            -\mathcal{H}_y^{n+\frac{1}{2}}\left(i-\frac{1}{2},j,k+\frac{1}{2}\right)
        \right]\\
        &-\frac{1}{\Delta y}\left[
            \mathcal{H}_x^{n+\frac{1}{2}}\left(i,j+\frac{1}{2},k+\frac{1}{2}\right)
            -\mathcal{H}_x^{n+\frac{1}{2}}\left(i,j-\frac{1}{2},k+\frac{1}{2}\right)
        \right]\\
        &-\mathcal{J}_z^{n+\frac{1}{2}}\left(i,j,k+\frac{1}{2}\right)
        \Bigg\}
    \end{aligned}
    \label{eq:Yee 三维时间步进公式-6}
\end{equation}
\par 其中 $\alpha(i,j)=\dfrac{\varepsilon}{\Delta t}-\dfrac{\sigma}{2}$,
$\beta(i,j)=\dfrac{\varepsilon}{\Delta t}+\dfrac{\sigma}{2}$。
\par 给定源电流、电场和磁场的初始值及边界条件,使用
式 (\ref{eq:Yee 三维时间步进公式-1}) 至
式 (\ref{eq:Yee 三维时间步进公式-3}) 计算下一个时间步的
磁场,然后用
式 (\ref{eq:Yee 三维时间步进公式-4}) 至
式 (\ref{eq:Yee 三维时间步进公式-6}) 计算下一个时间步的电
场。
\par 为了保证时间步进的稳定性,其时间步长应满足稳定性条件
\begin{equation}
    \Delta t \leq \frac{\sqrt{\mu \varepsilon}}
    {\sqrt{\frac{1}{(\Delta x)^2}+\frac{1}{(\Delta y)^2}+\frac{1}{(\Delta z)^2}}}
\end{equation}
\par 将数值色散误差公式拓展到三维情况,其表达式为
\begin{equation}
    \frac{\tilde{k}-k}{k}
    \approx\frac{1}{24}
    \Bigg\{
        \Big[
            (k\Delta x)^2\cos^4\phi_i
            +(k\Delta y)^2\sin^4\phi_i
        \Big]\sin^4\theta_i
        +(k\Delta z)^2\cos^4\theta_i
        -(\omega \Delta t)^2
    \Bigg\}
\end{equation}
\par 其中 $(\phi_i,\theta_i)$ 表示波的传播方向。
若选择 $\Delta x=\Delta y=\Delta z = h$ 和 $\Delta t = \frac{0.5h}{c}$,则数值相位误差为
\begin{equation}
    \begin{aligned}
        \frac{\tilde{k}-k}{k}
        &\approx\frac{(kh)^2}{24}
        \Bigg\{
            \Big[
                \cos^4\phi_i
                +\sin^4\phi_i
            \Big]\sin^4\theta_i
            +\cos^4\theta_i
            -\frac{1}{4}
        \Bigg\}\\
        &=\frac{\pi^2}{6}\left(\frac{h}{\lambda}\right)^2
        \Bigg\{
            \Big[
                \cos^4\phi_i
                +\sin^4\phi_i
            \Big]\sin^4\theta_i
            +\cos^4\theta_i
            -\frac{1}{4}
        \Bigg\}
    \end{aligned}
\end{equation}

\section{完美匹配层\cite{jin_theory_2015}}

\par 完美匹配层是一种通过理论上推导的人为设计的材料,可以设计
成对任意频率、任意极化和任意角度的平面波入射都完全吸收。

\subsection{频域分析}
\par 为了推导得到完美匹配层,
首先考虑无源情况下的修正 Maxwell 方程:
\begin{gather}
    \label{eq:corrected_curl_E}
    \nabla_s\times\bm{E}=-i\omega\mu\bm{H}\\
    \label{eq:corrected_curl_H}
    \nabla_s\times\bm{H}=i\omega\varepsilon\bm{E}\\ 
    \label{eq:corrected_div_E}
    \nabla_s\cdot(\varepsilon\bm{E})=0\\
    \label{eq:corrected_div_H}
    \nabla_s\cdot(\mu\bm{H})=0
\end{gather}
\par 其中 $\nabla_s$ 定义为
\begin{equation}
    \nabla_s=
    \bm{\hat{e}_x}\frac{1}{s_x}\frac{\partial}{\partial x}
    +\bm{\hat{e}_y}\frac{1}{s_y}\frac{\partial}{\partial y}
    +\bm{\hat{e}_z}\frac{1}{s_z}\frac{\partial}{\partial z}
    \label{eq:definition_of_nabla_s}
\end{equation}
\par $\nabla_s$ 可认为是 $x$、$y$ 和 $z$ 轴
分别被 $s_x$、$s_y$ 和 $s_z$ 因子拉伸的
的 $\nabla$ 算子。然后
考虑一个平面波,其电场和磁场分别为
\begin{align}
    \bm{E}&=\bm{E}_0e^{-i\bm{k}\cdot\bm{r}}
    =\bm{E}_0e^{-i(k_x x+k_y y+k_z z)}\\
    \bm{H}&=\bm{H}_0e^{-i\bm{k}\cdot\bm{r}}
    =\bm{H}_0e^{-i(k_x x+k_y y+k_z z)}
\end{align}
\par 将上述电场和磁场代入修正 Maxwell 方程组,得到
\begin{align}
    \label{eq:corrected_k_curl_E}
    \bm{k}_s\times\bm{E}&=\omega\mu\bm{H}\\
    \label{eq:corrected_k_curl_H}
    \bm{k}_s\times\bm{H}&=-\omega\varepsilon\bm{E}\\
    \label{eq:corrected_k_div_E}
    \bm{k}_s\cdot\bm{E}&=0\\
    \label{eq:corrected_k_div_H}
    \bm{k}_s\cdot\bm{H}&=0
\end{align}
\par 其中
\begin{equation}
    \bm{k}_s=\bm{\hat{e}_x}\frac{k_x}{s_x}
    +\bm{\hat{e}_y}\frac{k_y}{s_y}
    +\bm{\hat{e}_z}\frac{k_z}{s_z}
\end{equation}
\par 对式 \eqref{eq:corrected_k_curl_E}
两边叉乘 $\bm{k}_s$,得到
\begin{equation}
    \bm{k}_s\times(\bm{k}_s\times\bm{E})
    =\omega\mu\bm{k}_s\times\bm{H}
    =-\omega^2\mu\varepsilon\bm{E}
\end{equation}
\par 又由于 $\bm{k}_s\times(\bm{k}_s\times\bm{E})=
\bm{k}_s(\bm{k}_s\cdot\bm{E})
-(\bm{k}_s\cdot\bm{k}_s)\bm{E}$
以及 $\bm{k}_s\cdot\bm{E}=0$,上式变为
\begin{equation}
    (\bm{k}_s\cdot\bm{k}_s)\bm{E}
    =\omega^2\varepsilon\mu\bm{E}
\end{equation}
\par 由此得到色散关系式为
\begin{equation}
    \bm{k}_s\cdot\bm{k}_s
    =\omega^2\varepsilon\mu=k^2
\end{equation}
\par 代入 $\bm{k}_s$ 的表达式,得到
\begin{equation}
    \left(
        \frac{k_x}{s_x}
    \right)^2
    +\left(
        \frac{k_y}{s_y}
    \right)^2
    +\left(
        \frac{k_z}{s_z}
    \right)^2=k^2
\end{equation}
\par 满足修正 Maxwell 方程组的电磁波的色散关系为
\begin{equation}
    \left(
        \frac{k_x}{s_x}
    \right)^2
    +\left(
        \frac{k_y}{s_y}
    \right)^2
    +\left(
        \frac{k_z}{s_z}
    \right)^2=k^2
\end{equation}
\par 此方程的解为
\begin{equation}
    \left\{
        \begin{aligned}
            k_x&=k s_x\sin \theta \cos \phi\\
            k_y&=k s_y\sin \theta \sin \phi\\
            k_z&=k s_z\cos \theta
        \end{aligned}
    \right.
\end{equation}
\begin{figure}[htbp]
    \centering
    \includegraphics[width=0.4\textwidth]{./image/PlaneWave_PML.png}
    \caption{平面波入射示意图}
    \label{fig:PlaneWave_PML}
\end{figure}
\par 如图 \ref{fig:PlaneWave_PML} 所示,
考虑电磁波在拉伸坐标系中两个半空间分界面处的反射情况。
分界面与$xOy$平面
重合,对于$\text{TE}_z$ 入射的情况,入射波、反射波和
透射波的电场可分别写为:
\begin{align}
    \bm{E}_i&=\bm{\hat{e}_z} E_0 e^{-i\bm{k}_i\cdot\bm{r}}\\
    \bm{E}_r&=R_{\text{TE}}\bm{\hat{e}_z} E_0 
    e^{-i\bm{k}_r\cdot\bm{r}}\\
    \bm{E}_t&=T_{\text{TE}}\bm{\hat{e}_z} E_0 
    e^{-i\bm{k}_t\cdot\bm{r}}
\end{align}
\par 对于 $\text{TE}_z$ 入射的情况,
使用相位匹配条件及 $\bm{E}$ 与 $\bm{H}$ 的切向分量连续条件,可以得到
\begin{equation}
    R_{\text{TE}}
    =\frac{k_{1z}s_{2z}\mu_2-k_{2z}s_{1z}\mu_1}
    {k_{1z}s_{2z}\mu_2+k_{2z}s_{1z}\mu_1}
\end{equation}
\par 其中,下标 1 表示上半空间的媒质参数,
下标 2 表示下半空间的媒质参数。
类似地,可以得
到 $\text{TM}_z$ 入射时的反射系数为
\begin{equation}
    R_{\text{TM}}
    =\frac{k_{1z}s_{2z}\varepsilon_2-k_{2z}s_{1z}\varepsilon_1}
    {k_{1z}s_{2z}\varepsilon_2+k_{2z}s_{1z}\varepsilon_1}
\end{equation}
\par 若选择 $\varepsilon_1=\varepsilon_2$、
$\mu_1=\mu_2$、
$s_{1x}=s_{2x}$ 以及 $s_{1y}=s_{2y}$,可以得到
\begin{equation}
    R_{\text{TE}}=0
    \qquad
    R_{\text{TM}}=0
\end{equation}
上式在以下任何情形下均成立
\begin{itemize}
    \item 任意的 $s_{1z}$ 和 $s_{2z}$
    \item 任意的 $\theta$ 和 $\phi$
    \item 任意的频率
\end{itemize}
\par 由于位于 $xOy$ 平面的理想匹配分界面
与 $s_{1z}$ 和 $s_{2z}$ 无关,选择任意的 $s_{1z}$ 
和 $s_{2z}$ 均不会引起任何反射。
\par 如果选择 $s_{2z}=s'-js''$,
其中 $s'$ 和 $s''$ 是实数,
且有 $s'\geq1$和 $s''\geq0$,
则 $k_{2z}=k_2(s'-js'')\cos \theta$。
因此,透射波将在负 $z$ 方向指数衰减。
若我们将媒质 2 截断成厚度为 $L$ 的
介质层,并在其后放置一块导电平面,则其反射系数的幅度变为:
\begin{equation}
    |R(\theta)|=e^{-2k_2\cos \theta \int_{0}^{L}s''(z) \text{d}z}
\end{equation}
\begin{figure}[htbp]
    \centering
    \includegraphics[width=0.6\textwidth]{./image/Calculation_Domain.png}
    \caption{二维完美匹配层截断计算域示意图}
    \label{fig:Calculation_Domain}
\end{figure}
\par 以导电面为衬底的完美匹配层可用于时域有限差分仿真中的计算区域截断。
如图 \ref{fig:Calculation_Domain} 所示,
用以导电面为衬底的完美匹配层包裹感兴趣的计算区域。完美匹配
层区域介质的参数选择取决于具体的位置。

\subsection{时域分析}

\subsubsection{分裂场矢量法}

\par 首先要把式 \eqref{eq:corrected_curl_E} 
到 \eqref{eq:corrected_div_H} 的修正Maxwell方程组变换到时域,根据
式 \eqref{eq:definition_of_nabla_s} 可以得到
\begin{equation}
    \nabla_s\times\bm{E}
    =\frac{1}{s_x}\frac{\partial}{\partial x}
    \left(
        \bm{\hat{e}_x} \times\bm{E}
    \right)
    +\frac{1}{s_y}\frac{\partial}{\partial y}
    \left(
        \bm{\hat{e}_y} \times\bm{E}
    \right)
    +\frac{1}{s_z}\frac{\partial}{\partial z}
    \left(
        \bm{\hat{e}_z} \times\bm{E}
    \right)    
\end{equation}
\par 我们把磁场分裂为三个矢量分量:
\begin{equation}
    \bm{H}=\bm{H}_{sx}+\bm{H}_{sy}+\bm{H}_{sz}
\end{equation}
\par 其中
\begin{align}
    \label{eq:definition_of_H_sx}
    \frac{1}{s_x}\frac{\partial}{\partial x}
    \left(
        \bm{\hat{e}_x} \times\bm{E}
    \right)&=-i\omega\mu\bm{H}_{sx}\\
    \label{eq:definition_of_H_sy}
    \frac{1}{s_y}\frac{\partial}{\partial y}
    \left(
        \bm{\hat{e}_y} \times\bm{E}
    \right)&=-i\omega\mu\bm{H}_{sy}\\
    \label{eq:definition_of_H_sz}
    \frac{1}{s_z}\frac{\partial}{\partial z}
    \left(
        \bm{\hat{e}_z} \times\bm{E}
    \right)&=-i\omega\mu\bm{H}_{sz}
\end{align}
\par 选取 $s_x$ 、$s_y$ 和 $s_z$ 为
\begin{equation}
    s_x=1+\frac{\sigma_x}{i\omega\varepsilon}
    \qquad
    s_y=1+\frac{\sigma_y}{i\omega\varepsilon}
    \qquad
    s_z=1+\frac{\sigma_z}{i\omega\varepsilon}
    \label{eq:definition_of_sx_sy_sz}
\end{equation}
\par 将上述 $s_x$ 、$s_y$ 和 $s_z$代入
式 \eqref{eq:definition_of_H_sx} 至
\eqref{eq:definition_of_H_sz},并进行时域变换,
可以得到
\begin{align}
    \label{eq:time_domain_H_sx}
    \frac{\partial}{\partial x}
    \left(
        \bm{\hat{e}_x} \times\bm{\mathcal{E}}
    \right)
    &=-\mu\frac{\partial \bm{\mathcal{H}}_{sx}}{\partial t}
    -\frac{\sigma_x\mu}{\varepsilon} \bm{\mathcal{H}}_{sx}\\
    \label{eq:time_domain_H_sy}
    \frac{\partial}{\partial y}
    \left(
        \bm{\hat{e}_y} \times\bm{\mathcal{E}}
    \right)
    &=-\mu\frac{\partial \bm{\mathcal{H}}_{sy}}{\partial t}
    -\frac{\sigma_y\mu}{\varepsilon} \bm{\mathcal{H}}_{sy}\\
    \label{eq:time_domain_H_sz}
    \frac{\partial}{\partial z}
    \left(
        \bm{\hat{e}_z} \times\bm{\mathcal{E}}
    \right)
    &=-\mu\frac{\partial \bm{\mathcal{H}}_{sz}}{\partial t}
    -\frac{\sigma_z\mu}{\varepsilon} \bm{\mathcal{H}}_{sz}
\end{align}
\par 类似的,我们把电场分裂为三个矢量分量:
\begin{equation}
    \bm{E}=\bm{E}_{sx}+\bm{E}_{sy}+\bm{E}_{sz}
\end{equation}
\par 同样可以得到相应的时域方程为
\begin{align}
    \label{eq:time_domain_E_sx}
    \frac{\partial}{\partial x}
    \left(
        \bm{\hat{e}_x} \times\bm{\mathcal{H}}
    \right)
    &=\varepsilon\frac{\partial \bm{\mathcal{E}}_{sx}}{\partial t}
    +\sigma_x \bm{\mathcal{E}}_{sx}\\
    \label{eq:time_domain_E_sy}
    \frac{\partial}{\partial y}
    \left(
        \bm{\hat{e}_y} \times\bm{\mathcal{H}}
    \right)
    &=\varepsilon\frac{\partial \bm{\mathcal{E}}_{sy}}{\partial t}
    +\sigma_y \bm{\mathcal{E}}_{sy}\\
    \label{eq:time_domain_E_sz}
    \frac{\partial}{\partial z}
    \left(
        \bm{\hat{e}_z} \times\bm{\mathcal{H}}
    \right)
    &=\varepsilon\frac{\partial \bm{\mathcal{E}}_{sz}}{\partial t}
    +\sigma_z \bm{\mathcal{E}}_{sz}
\end{align}
\par 这些修正的时域Maxwell方程组可以使用Yee 网格进行离散。
现在考虑一个二维 $\text{TM}_z$ 问题,其中 
$\bm{\mathcal{E}}=\bm{\hat{e}_z}\mathcal{E}_z$,
$\bm{\mathcal{H}}=\bm{\hat{e}_x}\mathcal{H}_x + \bm{\hat{e}_y}\mathcal{H}_y$。
从式 \eqref{eq:time_domain_H_sx} 至
\eqref{eq:time_domain_E_sz} 中可得
$\bm{\mathcal{H}}_{sx}=\bm{\hat{e}_y}\mathcal{H}_y$,
$\bm{\mathcal{H}}_{sy}=\bm{\hat{e}_x}\mathcal{H}_x$ 和
$\bm{\mathcal{H}}_{sz}=0$,并且有
\begin{align}
    \frac{\partial \mathcal{E}_z}{\partial x}
    &=\mu\frac{\partial \mathcal{H}_{y}}{\partial t}
    +\frac{\sigma_x\mu}{\varepsilon} \mathcal{H}_{y}\\
    \frac{\partial \mathcal{E}_z}{\partial y}
    &=-\mu\frac{\partial \mathcal{H}_{x}}{\partial t}
    -\frac{\sigma_y\mu}{\varepsilon} \mathcal{H}_{x}
\end{align}
\par 同理可得 $\bm{\mathcal{E}}_{sx}=\bm{\hat{e}_z}\mathcal{E}_{sx,z}$,
$\bm{\mathcal{E}}_{sy}=\bm{\hat{e}_z}\mathcal{E}_{sy,z}$ 和
$\bm{\mathcal{E}}_{sz}=0$,其中 $\mathcal{E}_{z} = \mathcal{E}_{sx,z} + \mathcal{E}_{sy,z}$ 
并且有
\begin{align}
    \frac{\partial \mathcal{H}_y}{\partial x}
    &=\varepsilon\frac{\partial \mathcal{E}_{sx,z}}{\partial t}
    +\sigma_x \mathcal{E}_{sx,z}\\
    \frac{\partial \mathcal{H}_x}{\partial y}
    &=-\varepsilon\frac{\partial \mathcal{E}_{sy,z}}{\partial t}
    -\sigma_y \mathcal{E}_{sy,z}
\end{align}
\par 可以用 Yee 网格对上述方程进行离散化,从而得到
完美匹配层的时域有限差分更新公式。

\subsubsection{辅助微分方程法}

\par 除了分裂场矢量,另一种实现完美匹配层的方法是使用辅助矢量并求解对应的辅助微
分方程。完美匹配层等效于一种各向异性的色散媒质,其介电常数和磁导率张量分别为:
\begin{equation}
    \bm{\overline{\varepsilon}}=\varepsilon
    \begin{bmatrix}
        \dfrac{s_y s_z}{s_x} & 0 & 0\\
        0 & \dfrac{s_z s_x}{s_y} & 0\\
        0 & 0 & \dfrac{s_x s_y}{s_z}
    \end{bmatrix}
    \qquad
    \bm{\overline{\mu}}=\mu
    \begin{bmatrix}
        \dfrac{s_y s_z}{s_x} & 0 & 0\\
        0 & \dfrac{s_z s_x}{s_y} & 0\\
        0 & 0 & \dfrac{s_x s_y}{s_z}
    \end{bmatrix}
\end{equation}
\par 其中$\varepsilon$和$\mu$分别表示被
完美匹配层包围媒质的介电常数和磁导率,
而$s_x$、$s_y$,及$s_z$由
式 \eqref{eq:definition_of_sx_sy_sz} 给出。
在这样的媒质中,Maxwell方程组的前两个方程为:
\begin{gather}
    \label{eq:ADI_curl_E}
    \nabla\times\bm{E}=-i\omega
    \begin{bmatrix}
        s_y & 0 & 0\\
        0 & s_z & 0\\
        0 & 0 & s_x
    \end{bmatrix} \cdot \bm{B}\\
    \label{eq:ADI_curl_H}
    \nabla\times\bm{H}=i\omega
    \begin{bmatrix}
        s_y & 0 & 0\\
        0 & s_z & 0\\
        0 & 0 & s_x
    \end{bmatrix} \cdot \bm{D}
\end{gather}
\par 其中 $\bm{D}$ 和 $\bm{B}$ 分别为辅助矢量:
\begin{gather}
    \label{eq:definition_of_D}
    \bm{D} = \varepsilon
    \begin{bmatrix}
        \dfrac{s_z}{s_x} & 0 & 0\\
        0 & \dfrac{s_x}{s_y} & 0\\
        0 & 0 & \dfrac{s_y}{s_z}
    \end{bmatrix} \cdot \bm{E}\\
    \label{eq:definition_of_B}
    \bm{B} = \mu
    \begin{bmatrix}
        \dfrac{s_z}{s_x} & 0 & 0\\
        0 & \dfrac{s_x}{s_y} & 0\\
        0 & 0 & \dfrac{s_y}{s_z}
    \end{bmatrix} \cdot \bm{H}
\end{gather}
\par 以 $x$ 分量为例,式 \eqref{eq:ADI_curl_E} 和 \eqref{eq:ADI_curl_H}
以及式 \eqref{eq:definition_of_B} 和 \eqref{eq:definition_of_D}
可写为:
\begin{gather}
    \frac{\partial E_z}{\partial y}
    -\frac{\partial E_y}{\partial z}
    =-i\omega s_y B_x\\
    \frac{\partial H_z}{\partial y}
    -\frac{\partial H_y}{\partial z}
    =i\omega s_y D_x\\
    s_x D_x = \varepsilon s_z E_x\\
    s_x B_x = \mu s_z H_x
\end{gather}
\par 将 $s_x$,$s_y$ 和 $s_z$ 代入上述方程,并进行时域变换,可以得到:
\begin{gather}
    \frac{\partial \mathcal{E}_z}{\partial y}
    -\frac{\partial \mathcal{E}_y}{\partial z}
    =-\frac{\partial \mathcal{B}_x}{\partial t}
    -\frac{\sigma_y}{\varepsilon} \mathcal{B}_x\\
    \frac{\partial \mathcal{H}_z}{\partial y}
    -\frac{\partial \mathcal{H}_y}{\partial z}
    =\frac{\partial \mathcal{D}_x}{\partial t}
    +\frac{\sigma_y}{\varepsilon} \mathcal{D}_x\\
    \frac{\partial \mathcal{D}_x}{\partial t}
    +\frac{\sigma_x}{\varepsilon} \mathcal{D}_x
    =\varepsilon\frac{\partial \mathcal{E}_x}{\partial t}
    +\sigma_z \mathcal{E}_x\\
    \frac{\partial \mathcal{B}_x}{\partial t}
    +\frac{\sigma_x}{\varepsilon} \mathcal{B}_x
    =\mu\frac{\partial \mathcal{H}_x}{\partial t}
    +\frac{\sigma_z\mu}{\varepsilon} \mathcal{H}_x
\end{gather}

% ===============================================

\newpage
\printbibliography

\end{document}